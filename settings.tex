\makeatletter
\newcommand{\openbox}{\leavevmode
  \hbox to.77778em{%
  \hfil\vrule
  \vbox to.675em{\hrule width.6em\vfil\hrule}%
  \vrule\hfil}}
\gdef\proofSymbol{\openbox}
\newcommand{\proofname}{Proof}
\newcounter{proof}\newcounter{currproofctr}\newcounter{endproofctr}%
\newenvironment{proof}[1][\proofname]{
  \th@nonumberplain
  \def\theorem@headerfont{\itshape}%
  \normalfont
  \theoremsymbol{\ensuremath{_\blacksquare}}
  \@thm{proof}{proof}{{#1}.}}%
  {\@endtheorem}
\makeatother

\theoremstyle{plain}
\theoremheaderfont{\bfseries\itshape}
\theorembodyfont{\normalfont}

\newtheorem{lm}{Lemma}[chapter]
\newtheorem{ax}[lm]{Axiom}
\newtheorem{sz}[lm]{Satz}
\newtheorem{df}[lm]{Definition}
\newtheorem{kl}[lm]{Korollar}
\newtheorem{cd}[lm]{Quellcode}
\newtheorem{bs}[lm]{Beispiel}
\newtheorem{afg}[lm]{Aufgabe}
\newtheorem{al}[lm]{Algorithmus}

\setlength{\textwidth}{15cm}
\setlength{\textheight}{26cm}
\setlength{\topmargin}{-1.78cm}
\setlength{\evensidemargin}{13mm}
\setlength{\oddsidemargin}{-5mm}
\setlength{\headheight}{20pt}
\parskip 6pt plus 1pt minus 1pt
\parindent0pt
\pagestyle{fancy}

\fancyhf{}
\fancyfoot[OR]{\thepage}
\fancyfoot[EL]{\thepage}

\fancyhead[OR]{\rightmark}
\fancyhead[EL]{\leftmark}

\renewcommand{\headrulewidth}{0.5pt}
\renewcommand{\footrulewidth}{0.5pt}

\usepackage{color}
\usepackage{yfonts}

\makeatletter
\newcommand\mynobreakpar{\par\nobreak\@afterheading}
\makeatother

\newcounter{zahl}
\newcommand{\ag}{\stepcounter{zahl}\textbf{\thesection -\arabic{zahl}}}
\newcommand{\itema}{\item[\textit{\ag}]}
\newcommand{\itemas}{\item[$\star$\textit{\ag}]}
\newenvironment{tasks}{\setcounter{zahl}{0} \noindent\hrulefill\\\textbf{Übungen}\mynobreakpar\begin{enumerate}}{\end{enumerate}}
\newenvironment{tasksn}{\noindent\hrulefill\\\textbf{Übungen}\mynobreakpar\begin{enumerate}}{\end{enumerate}}

\newcommand{\sg}{\stepcounter{zahl}\textbf{\arabic{section}.\arabic{subsection}-\arabic{zahl}}}
\newcommand{\items}{\item[\textit{\sg}]}
\newcommand{\itemss}{\item[$\star$\textit{\sg}]}
\newenvironment{sols}{\setcounter{zahl}{0}\begin{enumerate}}{\end{enumerate}}

\newcommand{\emphpar}[1]{\emph{#1}\index{#1}}
\newcommand{\emphpart}[2]{\emph{#1}\index{#2!#1}}
\newcommand{\emphparr}[1]{\emph{#1}}

\newcommand{\f}[1][f]{#1(\cdot)}
\newcommand{\fs}[1][f]{#1{\scriptstyle(\cdot)}}
\newcommand{\D}[1][f]{D_{\fs[#1]}}
\newcommand{\W}[1][f]{W_{\fs[#1]}}
\newcommand{\Zi}[1][f]{Z_{\fs[#1]}}
\newcommand{\G}[1][f]{G_{\fs[#1]}}

\usepackage{twoopt}
\newcommandtwoopt{\nint}[4][a][b]{\int\limits_{#1}^{#2}\!\!#3\;\mathrm{d}#4}

\usepackage{titlesec}
\usepackage{tocloft}

\newcommand{\secmark}{}
\newcommand{\marktotoc}[1]{\renewcommand{\secmark}{#1}}
\newenvironment{advanced}{\renewcommand{\secmark}{$\star$\quad}%
\addtocontents{toc}{\protect\marktotoc{$\star$\quad}}}{\addtocontents{toc}{\protect\marktotoc{}}}

\titleformat{\chapter}
{\titlerule[3pt]
    \vspace{0.8ex}%
    \normalfont\Large\bfseries}{\thechapter}{20pt}{}
    [\vspace{1.2ex}%
    \titlerule]
\titleformat{\section}
    {\titlerule[1.5pt]
     \vspace{0.8ex}%
    \normalfont\large\bfseries}{\secmark\thesection}{1em}{}
\titleformat{\subsection}
    {\normalfont\normalsize\bfseries}{\secmark}{0em}{}
\titleformat{\subsubsection}
    {\normalfont\normalsize\bfseries}{\secmark}{0em}{}
\titleformat{\paragraph}[runin]
    {\normalfont\normalsize\bfseries}{}{0em}{}
\titleformat{\subparagraph}[runin]
    {\normalfont\normalsize\bfseries}{}{0em}{}

\titlespacing*{\chapter}      {0pt}{50pt}{40pt}
\titlespacing*{\section}      {0pt}{3.5ex plus 1ex minus 0.2ex}{2.3ex plus 0.2ex}
\titlespacing*{\subsection}   {0pt}{3.25ex plus 1ex minus 0.2ex}{1.5ex plus 0.2ex}
\titlespacing*{\subsubsection}{0pt}{3.25ex plus 1ex minus 0.2ex}{1.5ex plus 0.2ex}
\titlespacing*{\paragraph}   {0pt}{3.25ex plus 1ex minus 0.2ex}{1em}
\titlespacing*{\subparagraph} {\parindent}{3.25ex plus 1ex minus 0.2ex}{1em}

\newcommand{\chapters}[1]{\begin{advanced}\chapter{#1}\end{advanced}}
\newcommand{\sections}[1]{\begin{advanced}\section{#1}\end{advanced}}
\newcommand{\subsections}[1]{\begin{advanced}\subsection{#1}\end{advanced}}
\newcommand{\subsubsections}[1]{\begin{advanced}\subsubsection{#1}\end{advanced}}

